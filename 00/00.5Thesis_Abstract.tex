%%%%%%%%%%%%%%%%%%%%%%%%%%%%%%%%%%%%%%%%%%%%%%%%%%%%%%%%%%%%%%%%%%%%%%%%
%                                                                      %
%     File: Thesis_Abstract.tex                                        %
%     Tex Master: Thesis.tex                                           %
%                                                                      %
%     Author: Israel Sother                                            %
%     Last modified: 27 May 2024                                       %
%                                                                      %
%%%%%%%%%%%%%%%%%%%%%%%%%%%%%%%%%%%%%%%%%%%%%%%%%%%%%%%%%%%%%%%%%%%%%%%%

\section*{Abstract}

% Add entry in the table of contents as section
\addcontentsline{toc}{section}{Abstract}

This work presents the development of a control strategy for the motors of the Formula Student Team of Instituto Superior Técnico. The control strategy was developed with the main goal of improving the dynamic response of the motor torque and the system efficiency. The motor was experimentally characterized and a simulation environment was created to test different control strategies. The explicit \gls{mpc} was selected as the most suitable strategy due to its fast response and low current ripple. A novel horizon extention technique was proposed to reduce model mismatch, which gave the name of the controller: \acrfull{rush}. The control strategy was implemented in an \gls{fpga} and a test bench setup was developed. The experimental results showed a close match with the simulation results, validating the torque estimation, the current measurements, and the control strategy. The system efficiency was improved due to the reduction in current \gls{thd} and the torque response was improved by orders of magnitude compared to the current solution.
\vfill

\textbf{\Large Keywords:} Explicit MPC, Motor Control, Electric Vehicle, Formula Student, Horizon Extension